\section{Método de los momentos}
El método de los momentos es un método intuitivo para estimar parámetros cuando otros métodos, más atractivos, resultan ser muy difíciles de calcular explícitamente. Este método puede ser usado para calcular un punto de partida para algoritmos de aproximación. 

\begin{definition} [Método de los Momentos]
Sean $X_1,...,X_n$ un muestreo aleatorio simple de una distribución que está indexada por un parámetro k-dimensional $\theta$, y que tiene por lo menos $k$ momentos finitos.  Para $j=1,...,k$ definimos $\mu_{j}(\theta)=\mathbb{E}(X_{1}^{j}|\theta)$. Supongamos que la función $\mu(\theta)= (\mu_1 (\theta),...,\mu_k(\theta)$ es inyectiva en $\theta$, y sea $M(\mu_1, . . . , \mu_k)$ su inversa. \\
Definamos los momentos por: 
$$
m_{j}=\dfrac{1}{n} \sum_{i=1}^{n} X_{i}^{j}
$$
con $j=1,...,k$. El estimador del método de los momentos de $\theta$ es 
$M(m_1,...,m_k)$. 
\end{definition}

\begin{example}

\end{example}