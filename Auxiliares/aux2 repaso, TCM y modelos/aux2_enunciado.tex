\documentclass[letterpaper,11pt]{article}
\oddsidemargin -1.0cm \textwidth 16.5cm
%\usepackage[latin1]{inputenc}
%\usepackage[spanish]{babel}

\usepackage[T1]{fontenc}
\usepackage[utf8]{inputenc}
\usepackage[spanish]{babel}
\usepackage{amsfonts,setspace}
\usepackage{amsmath}
\usepackage{amssymb, amsmath, amsthm}
\usepackage{comment}
%\usepackage[utf8]{inputenc}
\usepackage{amssymb}
\usepackage{bbm}
\usepackage{dsfont}
\usepackage{anysize}
\usepackage{multicol}
\usepackage{enumerate}
\usepackage{enumitem}
\usepackage{graphicx}
\usepackage{subfigure}
\usepackage{float}

\usepackage[font=small,labelfont=bf]{caption}
\usepackage[left=2cm,top=2cm,right=2cm, bottom=2cm]{geometry}
\usepackage{fancyhdr}
\pagestyle{fancy}

%Teoremas, Lemas, etc.
\theoremstyle{plain}
\newtheorem{teo}{Teorema}
\newtheorem{lem}{Lemma}
\newtheorem{prop}{Proposici\'on}
\newtheorem{cor}{Corolario}
\theoremstyle{definition}
\newtheorem{defi}{Definici\'on}
\newtheorem{eje}{Ejemplo}
\newtheorem{obs}{Observaci\'on}
\newcommand{\norm}[1]{\lVert#1\rVert}
\newcommand{\R}{\mathbb{R}}
\newcommand{\N}{\mathbb{N}}
\newcommand{\tiende}{\rightarrow}
\newcommand{\borel}{\mathcal{B}}
\newcommand{\proba}{\mathbb{P}}
\newcommand{\ds}{\displaystyle}
\newcommand{\partes}{\mathcal{P}}
\newcommand{\E}{\mathbb{E}}
\newcommand{\Var}{\mathbb{V}ar}
\newcommand{\1}{\mathbbm{1}}
% fin macros


%\fancypagestyle{plain}{%
%\fancyhf{}
%\lhead{\footnotesize\itshape\bfseries\rightmark}
%\rhead{\footnotesize\itshape\bfseries\leftmark}}
%\pagestyle{plain}


%Inicializando documento1

\begin{document}

%Encabezado
\fancyhead[L]{\itshape{Facultad de Ciencias F\'isicas y Matem\'aticas}}
\fancyhead[R]{\itshape{Universidad de Chile}}


\begin{minipage}{11.5 cm}
\begin{flushleft}
\hspace*{-0.6cm}\textbf{MA3402-01 Estadística}\\
\hspace*{-0.6cm}\large Profesor: Felipe Tobar\\
\hspace*{-0.6cm}\large Auxiliares: Diego Marchant D., Francisco Vásquez L.\\

%\hspace*{-0.6cm}\textit{``Un matem\'atico que no tenga algo de poeta jam\'as ser\'a un completo matem\'atico''} - Karl Weierstrass\\

\end{flushleft}\end{minipage}


\begin{picture}(2,3)
\put(400,-10){\includegraphics[scale=0.17]{./dim.pdf}}
\end{picture}

%Titulo
\begin{center}
\LARGE\textbf{Auxiliar 2: Modelos Paramétricos, TCL y repaso}\\

\vspace{0.4cm}
\large 8 de Agosto de 2019
\end{center}

\textbf{Modelo de Poisson}\\

\begin{enumerate}[label=\textbf{P{\arabic*}}]

\item Dada una muestra aleatoria simple $X = (X_1, X_2,  \ldots, X_n)$ (es decir, iid) de un modelo Poisson de parámetros $\lambda >0$, nos interesa estimar la probabilidad del valor $0$, es decir, $g(\lambda) = \proba _\lambda (X_1 = 0)  =e^{-\lambda}$.\\

Para esto, consideremos los siguientes tres estimadores:

$$\hat{g}_1 (X) = e^{-\bar{X}}, \qquad \hat{g}_2 (X) = \frac{1}{n}\sum_{i=1}^{n}\mathds{1}_{\lbrace X_i =0\rbrace}, \qquad \hat{g}_3 (X) = \left(1- \frac{1}{n} \right)^{n\bar{X}}$$

¿Cuánto es el sesgo en cada uno de ellos?

\textbf{TCL}

\item Ahora veamos el error en la estimación del parámetro $\lambda$. Para esto proponga un estimador insesgado de este parámetro y con ayuda del TCL vea cómo se comporta asintóticamente el estimador en función del error gaussiano. \textbf{Indicación:} Estudie el comportamiento de la familia de variables aleatorias $\ds Z_n = \frac{S_n - n\E(X_1)}{\sqrt{\Var(X_1)}}$ donde $\ds S_n = \sum_{i=1}^n X_i$

\item Ahora veamos el comportamiento numérico asociado a este teorema. Para ello:

\begin{itemize}
    
    \item Cree una función que genere M muestras de la distribución $Z_n$
    
    \item Para los valores $n = 10, 50, 100, 200, 500, 1000$ grafique un histograma de las M muestras (puede tomar, por ejemplo, M = 1000)
    
    
    
\end{itemize}

\end{enumerate}

\end{document}
