\documentclass[11pt]{article}
\usepackage{graphicx}
\usepackage{cancel}
\usepackage{fancyhdr}
\usepackage{graphicx}
\usepackage{enumerate}
\usepackage{multicol}
\usepackage{mwe} % for blindtext and example-image-a in example
\usepackage{wrapfig}
\usepackage{graphicx}
\usepackage{xcolor}
\usepackage{dsfont}


\usepackage{amsmath,amssymb}
\usepackage[utf8]{inputenc}


\renewcommand{\labelenumi}{\normalsize\bfseries P\arabic{enumi}.}
\renewcommand{\labelenumii}{\normalsize\bfseries (\alph{enumii})}
\renewcommand{\labelenumiii}{\normalsize\bfseries \roman{enumiii})}

\oddsidemargin -1.0cm
\textwidth 18.4cm
\topmargin -2.5cm
\textheight 24cm

\DeclareMathOperator{\sen}{sen}
\DeclareMathOperator{\senh}{senh}
\DeclareMathOperator{\arcsen}{arcsen}
\DeclareMathOperator{\tg}{tg}
\DeclareMathOperator{\arctg}{arctg}
\DeclareMathOperator{\ctg}{ctg}
\DeclareMathOperator{\dom}{Dom}
\DeclareMathOperator{\sech}{sech}
\DeclareMathOperator{\rec}{Rec}
\DeclareMathOperator{\inte}{Int}
\DeclareMathOperator{\adh}{Adh}
\DeclareMathOperator{\fr}{Fr}
\DeclareMathOperator{\Ima}{Im}
\DeclareMathOperator{\dist}{dist}
\DeclareMathOperator{\argmin}{\text{argmín}}

\let\lim=\undefined\DeclareMathOperator*{\lim}{\text{lím}}
\let\max=\undefined\DeclareMathOperator*{\max}{\text{máx}}
\let\min=\undefined\DeclareMathOperator*{\min}{\text{mín}}
\let\inf=\undefined\DeclareMathOperator*{\inf}{\text{ínf}}

\providecommand{\abs}[1]{\lvert #1\rvert}


\newcommand{\ssi}{\Longleftrightarrow}
\newcommand{\imp}{\Longrightarrow}
\newcommand{\pmi}{\Longleftarrow}
\newcommand{\ipartial}[2]{\dfrac{\partial #1}{\partial #2}}
\newcommand{\ider}[2]{\dfrac{d #1}{d #2}}
\newcommand{\iipartial}[2]{\dfrac{\partial^2 #1}{\partial #2^2}}
\newcommand{\iider}[2]{\dfrac{d^2 #1}{d #2^2}}
\newcommand{\ijpartial}[3]{\dfrac{\partial^2 #1}{\partial #2 \partial #3}}
\newcommand{\N}{\mathbb{N}}
\newcommand{\Z}{\mathbb{Z}}
\newcommand{\C}{\mathbb{C}}
\newcommand{\Q}{\mathbb{Q}}
\newcommand{\R}{\mathbb{R}}
\newcommand{\K}{\mathbb{K}}
\newcommand{\sol}{\textbf{\emph{Solución: }}}
\newcommand{\dem}{\textbf{\emph{Demostración: }}}
\newcommand{\aux}[4]{\Large \textbf{Auxiliar #1: #2}\\ \normalsize \textbf{Profesor: } #3\\ \textbf{Auxiliares: }#4}
\newcommand{\pauta}[4]{\Large \textbf{Pauta #1 N#2}\\ \normalsize \textbf{Profesor: }#3\\ \textbf{Auxiliares: }#4}
\newcommand{\enc}[3]{\Large \textbf{#1}\\ \normalsize \textbf{Profesor: } #2\\ \textbf{\Auxiliares :} #3}

\pagestyle{fancy}
\usepackage[left=2cm,top=2cm,right=2cm]{geometry}

\begin{document}


\fancyhead[L]{{Facultad de Ciencias Físicas y Matemáticas}}
\fancyhead[R]{{Universidad de Chile}}
\begin{wrapfigure}{R}{0.3\textwidth} %this figure will be at the right
    \vspace{-5mm}
    \includegraphics[width=0.35\textwidth]{dim.pdf}
\end{wrapfigure}
\noindent Departamento de Ingeniería Matemática\\ 
MA3402-1 Estadística\\ 
14 de agosto de 2019

\hfill\break
\begin{center}
\aux{3}{Estadísticos y Suficiencia}{Felipe Tobar}{Diego Marchant, Francisco Vásquez}
\end{center}

\begin{enumerate}\setlength{\itemsep}{0.4cm}

\item \textbf{Estadísticos y suficiencia}
    
    \begin{itemize}
    \item[(i)] ¿Qué es un estadístico?, ¿Qué es un estadístico suficiente?
    \item[(ii)] Ejemplos
    \item[(iii)] Estudiemos el problema de estimación del área de un rectángulo. Consideremos un rectángulo de lados a,b desconocidos. Sea $X=(X_1,...,X_n)$ una MAS que representa el lado a) del rectángulo con $X_i\sim \mathcal{N}(a,\sigma^2)$ e $Y=(Y_1,...,Y_n)$ otra MAS que representa el lado b) del rectángulo con $Y_i\sim \mathcal{N}(b,\sigma^2)$, $\sigma$ conocido y X e Y independientes. Plantee el modelo paramétrico asociado y encuentre un estadístico suficiente. 
    \end{itemize}

\item Sea $X=(X_1,...,X_n)$ una MAS con $X_i\sim U(\alpha,\beta)$ donde $\alpha,\beta$ son desconocidos. Demuestre que $T(X)=\left(\min\limits_{i=1,...,n}{X_i},\max\limits_{i=1,...,n}{X_i}\right)$ es un estadístico suficiente para $(\alpha,\beta)$.

\item Sea $X=(X_1,...,X_n)$ una MAS con $X_i\sim \Gamma(\alpha,\beta)$ la distribución Gamma dada por:
\begin{equation}
    \nonumber 
    \Gamma(\alpha,\beta)\sim \frac{1}{\Gamma(\alpha)\beta^\alpha}x^{\alpha-1}e^{-\frac{1}{\beta}x}
\end{equation}
Demuestre que $T(X)=\left (\prod\limits_{i=1}^{n}x_i,\sum\limits_{i=1}^{n}x_i  \right )$ es un estadistico suficiente para $(\alpha,\beta)$

\item Sea $X=(X_1,...,X_n)$ una MAS con función de densidad dada por:
\begin{equation}
    \nonumber 
    f(x\mid \theta)=\frac{\theta}{(1+x)^{1+\theta}} , 0<x<\infty,\theta>0
\end{equation}
Con $\theta$ desconocido. Encuentre un estadístico suficiente.

\end{enumerate}

\end{document}












\grid
