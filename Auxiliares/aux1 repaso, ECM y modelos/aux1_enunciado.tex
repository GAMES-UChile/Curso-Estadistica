\documentclass[11pt]{article}
\usepackage{graphicx}
\usepackage{cancel}
\usepackage{fancyhdr}
\usepackage{graphicx}
\usepackage{enumerate}
\usepackage{multicol}
\usepackage{mwe} % for blindtext and example-image-a in example
\usepackage{wrapfig}
\usepackage{graphicx}
\usepackage{xcolor}
\usepackage{dsfont}


\usepackage{amsmath,amssymb}
\usepackage[utf8]{inputenc}


\renewcommand{\labelenumi}{\normalsize\bfseries P\arabic{enumi}.}
\renewcommand{\labelenumii}{\normalsize\bfseries (\alph{enumii})}
\renewcommand{\labelenumiii}{\normalsize\bfseries \roman{enumiii})}

\oddsidemargin -1.0cm
\textwidth 18.4cm
\topmargin -2.5cm
\textheight 24cm

\DeclareMathOperator{\sen}{sen}
\DeclareMathOperator{\senh}{senh}
\DeclareMathOperator{\arcsen}{arcsen}
\DeclareMathOperator{\tg}{tg}
\DeclareMathOperator{\arctg}{arctg}
\DeclareMathOperator{\ctg}{ctg}
\DeclareMathOperator{\dom}{Dom}
\DeclareMathOperator{\sech}{sech}
\DeclareMathOperator{\rec}{Rec}
\DeclareMathOperator{\inte}{Int}
\DeclareMathOperator{\adh}{Adh}
\DeclareMathOperator{\fr}{Fr}
\DeclareMathOperator{\Ima}{Im}
\DeclareMathOperator{\dist}{dist}
\DeclareMathOperator{\argmin}{\text{argmín}}

\let\lim=\undefined\DeclareMathOperator*{\lim}{\text{lím}}
\let\max=\undefined\DeclareMathOperator*{\max}{\text{máx}}
\let\min=\undefined\DeclareMathOperator*{\min}{\text{mín}}
\let\inf=\undefined\DeclareMathOperator*{\inf}{\text{ínf}}

\providecommand{\abs}[1]{\lvert #1\rvert}


\newcommand{\ssi}{\Longleftrightarrow}
\newcommand{\imp}{\Longrightarrow}
\newcommand{\pmi}{\Longleftarrow}
\newcommand{\ipartial}[2]{\dfrac{\partial #1}{\partial #2}}
\newcommand{\ider}[2]{\dfrac{d #1}{d #2}}
\newcommand{\iipartial}[2]{\dfrac{\partial^2 #1}{\partial #2^2}}
\newcommand{\iider}[2]{\dfrac{d^2 #1}{d #2^2}}
\newcommand{\ijpartial}[3]{\dfrac{\partial^2 #1}{\partial #2 \partial #3}}
\newcommand{\N}{\mathbb{N}}
\newcommand{\Z}{\mathbb{Z}}
\newcommand{\C}{\mathbb{C}}
\newcommand{\Q}{\mathbb{Q}}
\newcommand{\R}{\mathbb{R}}
\newcommand{\K}{\mathbb{K}}
\newcommand{\sol}{\textbf{\emph{Solución: }}}
\newcommand{\dem}{\textbf{\emph{Demostración: }}}
\newcommand{\aux}[4]{\Large \textbf{Auxiliar #1: #2}\\ \normalsize \textbf{Profesor: } #3\\ \textbf{Auxiliares: }#4}
\newcommand{\pauta}[4]{\Large \textbf{Pauta #1 N#2}\\ \normalsize \textbf{Profesor: }#3\\ \textbf{Auxiliares: }#4}
\newcommand{\enc}[3]{\Large \textbf{#1}\\ \normalsize \textbf{Profesor: } #2\\ \textbf{\Auxiliares :} #3}

\pagestyle{fancy}
\usepackage[left=2cm,top=2cm,right=2cm]{geometry}

\begin{document}


\fancyhead[L]{{Facultad de Ciencias Físicas y Matemáticas}}
\fancyhead[R]{{Universidad de Chile}}
\begin{wrapfigure}{R}{0.3\textwidth} %this figure will be at the right
    \vspace{-5mm}
    \includegraphics[width=0.35\textwidth]{dim.pdf}
\end{wrapfigure}
\noindent Departamento de Ingeniería Matemática\\ 
MA3402-1 Estadística\\ 
07 de agosto de 2019

\hfill\break
\begin{center}
\aux{1}{Modelos paramétricos, ECM y Recuerdos}{Felipe Tobar}{Diego Marchant, Francisco Vásquez}
\end{center}

\begin{enumerate}\setlength{\itemsep}{0.4cm}

\item Estudios relacionados con el comportamiento de ciertos bichos indican que estos tienden a organizarse al azar, linealmente, en un intervalo de longitud $\theta > 0$, a la derecha de un punto donde se ubica una feromona. Nos gustaría estimar el valor del parámetro $\theta$. Sea $X=(X_1,...,X_n)$ una muestra aleatoria simple (MAS) de la distancia de n bichos con respecto a la feromona.

\begin{enumerate}
    \item Defina el modelo paramétrico correspondiente.
    \item Considere el estimador $\hat{\theta}=2\Bar{X_n}$. ¿Será insesgado? Si no lo es, modifíquelo para que lo sea.
    \item Ahora, considere el estimador $\hat{\theta}=\max\{ X_1,...,X_n\}$. ¿Será insesgado? Si no lo es, modifíquelo para que lo sea.
    \item Calcule el ECM para cada uno de los estimadores y compárelos.
\end{enumerate}

\item Sea una muestra aleatoria simple (MAS) $X=(X_1,...,X_n)$ dada por $X_i\sim\mathcal{N}(\mu,\sigma^2),\forall i=1,...,n$. Con $\mu$ y $\sigma$ parámetros desconocidos.

Considere
\[ S^2:= \frac{1}{n-1}\sum\limits_{i=1}^{n}(X_i-\Bar{X}_n)^2 \]
Donde $\Bar{X}_n$ es el promedio de X. Muestre que $S^2$ es insesgado como estimador de $\sigma^2$ y calcule su varianza.

Considere
\[ \hat{\sigma}^2:= \frac{1}{n}\sum\limits_{i=1}^{n}(X_i-\Bar{X}_n)^2 \]

Muestre que $\hat{\sigma}^2$ es sesgado como estimador de $\sigma^2$, pero es asintoticamente insesgado.

Calcule su error cuadrático medio y concluya que:

\[ECM(\hat{\sigma}^2)=\mathbb{E}((\hat{\sigma}^2-\sigma^2)^2)<\mathbb{E}((S^2-\sigma^2)^2)=Var(S^2)=ECM(S^2)\]

Considere
\[\hat{\sigma}^2_\rho:=\rho S^2   \]
con $0<\rho \in \mathbb{R}$ fijo

Calcule su error cuadrático medio y encuentre $\rho^\star$ tal que:

\[ ECM(\hat{\sigma}^2_{\rho^\star})=\inf_{\rho>0} ECM(\hat{\sigma}^2_\rho)\]

Muestre que $\hat{\sigma}^2_{\rho^\star}$ es sesgado como estimador de $\sigma^2$, pero es asintoticamente insesgado.




\end{enumerate}

\end{document}












\grid
