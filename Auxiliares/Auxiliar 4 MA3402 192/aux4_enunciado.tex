\documentclass[letterpaper,11pt]{article}
\oddsidemargin -1.0cm \textwidth 16.5cm
%\usepackage[latin1]{inputenc}
%\usepackage[spanish]{babel}

\usepackage[T1]{fontenc}
\usepackage[utf8]{inputenc}
\usepackage[spanish]{babel}
\usepackage{amsfonts,setspace}
\usepackage{amsmath}
\usepackage{amssymb, amsmath, amsthm}
\usepackage{comment}
%\usepackage[utf8]{inputenc}
\usepackage{amssymb}
\usepackage{bbm}
\usepackage{dsfont}
\usepackage{anysize}
\usepackage{multicol}
\usepackage{enumerate}
\usepackage{enumitem}
\usepackage{graphicx}
\usepackage{subfigure}
\usepackage{float}

\usepackage[font=small,labelfont=bf]{caption}
\usepackage[left=2cm,top=2cm,right=2cm, bottom=2cm]{geometry}
\usepackage{fancyhdr}
\pagestyle{fancy}

%Teoremas, Lemas, etc.
\theoremstyle{plain}
\newtheorem{teo}{Teorema}
\newtheorem{lem}{Lemma}
\newtheorem{prop}{Proposici\'on}
\newtheorem{cor}{Corolario}
\theoremstyle{definition}
\newtheorem{defi}{Definici\'on}
\newtheorem{eje}{Ejemplo}
\newtheorem{obs}{Observaci\'on}
\newcommand{\norm}[1]{\lVert#1\rVert}
\newcommand{\R}{\mathbb{R}}
\newcommand{\N}{\mathbb{N}}
\newcommand{\tiende}{\rightarrow}
\newcommand{\borel}{\mathcal{B}}
\newcommand{\proba}{\mathbb{P}}
\newcommand{\ds}{\displaystyle}
\newcommand{\partes}{\mathcal{P}}
\newcommand{\E}{\mathbb{E}}
\newcommand{\Var}{\mathbb{V}ar}
\newcommand{\1}{\mathbbm{1}}
% fin macros


%\fancypagestyle{plain}{%
%\fancyhf{}
%\lhead{\footnotesize\itshape\bfseries\rightmark}
%\rhead{\footnotesize\itshape\bfseries\leftmark}}
%\pagestyle{plain}


%Inicializando documento1

\begin{document}

%Encabezado
\fancyhead[L]{\itshape{Facultad de Ciencias F\'isicas y Matem\'aticas}}
\fancyhead[R]{\itshape{Universidad de Chile}}


\begin{minipage}{11.5 cm}
\begin{flushleft}
\hspace*{-0.6cm}\textbf{MA3402-01 Estadística}\\
\hspace*{-0.6cm}\large Profesor: Felipe Tobar\\
\hspace*{-0.6cm}\large Auxiliares: Diego Marchant D., Francisco Vásquez L.\\

%\hspace*{-0.6cm}\textit{``Un matem\'atico que no tenga algo de poeta jam\'as ser\'a un completo matem\'atico''} - Karl Weierstrass\\

\end{flushleft}\end{minipage}


\begin{picture}(2,3)
\put(400,-10){\includegraphics[scale=0.17]{./fcfm.pdf}}
\end{picture}

%Titulo
\begin{center}
\LARGE\textbf{Auxiliar 4: Neyman-Fisher y la familia paramétrica}\\

\vspace{0.4cm}
\large 21 de Agosto de 2019
\end{center}

En todo el contexto del auxiliar, $X = (X_1, X_2, \ldots, X_n)$ representa una muestra aleatoria simple del modelo paramétrico en que se esté trabajando.

\begin{enumerate}[label=\textbf{P{\arabic*}}]


\item Encuentre un estadístico suficiente para la familia paramétrica binomial 
$$\mathcal{P} = \left\{ \proba_{p} \ | \ \proba(X=x) = \binom{n}{x}p^x(1-p)^{n-x} \right\}$$

\item Encuentre un estadístico suficiente para la familia paramétrica de Poisson 
$$\mathcal{P} = \left\{ \proba_{\lambda} \ | \ \proba(X=k) = \frac{e^{-\lambda}\lambda^k}{k!}\right\}$$

\item Muestre que el estadístico $\ds T(X) = \sum_{i=1}^n X_i$ es minimal suficiente para la familia paramétrica exponencial

$$\mathcal{P} = \left\{ \proba_{\theta} \ | \ f_{X_i} (x) = \theta e^{-\theta x}\right\}$$

\end{enumerate}

\end{document}
