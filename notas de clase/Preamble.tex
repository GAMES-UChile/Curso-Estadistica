%preamble para fuente, creo que Palatino es mejor para libros (FT)
\usepackage{mathpazo} % math & rm
\linespread{1.05}        % Palatino needs more leading (space between lines)
\usepackage[scaled]{helvet} % ss
\usepackage{courier} % tt
\normalfont
\usepackage[T1]{fontenc}

%Libro es en español (FT)
\usepackage[utf8]{inputenc}
\usepackage[spanish]{babel}
\usepackage{pgfplots}
\pgfplotsset{compat=1.6}

%tamaño del libro con marcas para reducir el tamaño carte (FT)
\usepackage{float}
\usepackage{geometry}
\usepackage[font=small,labelfont=bf]{caption}
%\geometry{layoutheight=230mm,layoutwidth=160mm,layoutvoffset=30mm,layouthoffset=20mm,showcrop}
\usepackage{csquotes}


% Agregado por FVasquez
%\usepackage{tocloft}
\usepackage{setspace}

\usepackage{cancel}
\usepackage{todonotes}
\usepackage{tcolorbox}
\usepackage{hyperref}
\usepackage{pgfplots}
\usepackage{tikz}
\usepackage{pgfplots}
\usepackage{mathtools,amssymb}

\renewcommand{\numberline}[1]{#1.~}

%listing package para código
\usepackage{listings}
\usepackage{xcolor}
 
\definecolor{codegreen}{rgb}{0,0.6,0}
\definecolor{codegray}{rgb}{0.5,0.5,0.5}
\definecolor{codepurple}{rgb}{0.58,0,0.82}
\definecolor{backcolour}{rgb}{0.95,0.95,0.92}
 
\lstdefinestyle{mystyle}{
	xleftmargin=0.15\textwidth,
	linewidth=0.8\textwidth,
    backgroundcolor=\color{backcolour},   
    commentstyle=\color{codegreen},
    keywordstyle=\color{magenta},
    numberstyle=\tiny\color{codegray},
    stringstyle=\color{codepurple},
    basicstyle=\ttfamily\footnotesize,
    breakatwhitespace=true,         
    breaklines=true,                 
    captionpos=b,                    
    keepspaces=true,                 
    numbers=left,                    
    numbersep=5pt,                  
    showspaces=false,                
    showstringspaces=false,
    showtabs=false,                  
    tabsize=2
}
 
\lstset{style=mystyle}


\usepackage{amsmath}

\renewcommand\qedsymbol{$\blacksquare$}

 

%theorems
\usepackage{amsthm}
% Teoremas

% Agregada por FVasquez
\newtheoremstyle{mytheoremstyle} % name
    {8pt}                % Space above
    {8pt}                % Space below
    {}                   % Body font
    {20pt}                   % Indent amount
    {\bfseries}          % Theorem head font
    {:}                  % Punctuation after theorem head
    {10pt}               % Space after theorem head
    {}                   % Theorem head spec
% Agregada por FVasquez
\theoremstyle{mytheoremstyle}

\newtheorem{definition}{Definición}[chapter]
\newtheorem{example}{Ejemplo}[chapter]
\newtheorem{exercise}{Ejercicio}[chapter]
\newtheorem{theorem}{Teorema}[chapter]
\newtheorem{lemma}{Lema}[chapter]
\newtheorem{remark}{Observación}[chapter]

%macros
\newcommand*\clase[1]{\vspace{2em}{\color{red}\par\noindent\raisebox{.8ex}{\makebox[\linewidth]{\hrulefill\hspace{1ex}\raisebox{-.8ex}{#1}\hspace{1ex}\hrulefill}}\vspace{0em}}}

\usepackage{color}
\providecommand{\red}[1]{\textcolor{red}{{\bf #1}}}


% Agregado por FVasquez
\usepackage{titlesec}
\titleformat*{\section}{\LARGE\bfseries}
\titleformat*{\subsection}{\Large\bfseries}
\titleformat*{\subsubsection}{\large\bfseries}
\titleformat*{\paragraph}{\large\bfseries}
\titleformat*{\subparagraph}{\large\bfseries}
\titleformat{\chapter}[display]
  {\normalfont\huge\bfseries\filcenter}{Capítulo \ \thechapter}{20pt}{\Huge}
\titlespacing*{\chapter}
  {0pt}{30pt}{20pt}

\def\familiaparametrica{\mathcal{P}  = \{P_\theta\ \tq\ \theta\in\Theta\}}
\def\familiaparametricaO{\mathcal{P}  = \{P_\theta\ \tq\ \theta\in\Omega\}}
\def\densidadparametrica{\{p_\theta\ \tq\ \theta\in\Theta\}}
\newcommand{\E}[1]{\mathbb{E} \left(#1\right)}
\newcommand{\V}[1]{\mathbb{V} \left(#1\right)}
\newcommand{\Prob}[1]{\mathbb{P} \left(#1\right)}
\newcommand{\Et}[1]{\mathbb{E}_\theta \left(#1\right)}
\newcommand{\Vt}[1]{\mathbb{V}_\theta \left(#1\right)}
\newcommand{\Probt}[1]{\mathbb{P}_\theta \left(#1\right)}
\newcommand{\Probtu}[1]{\mathbb{P}_{\theta_1} \left(#1\right)}
\newcommand{\Probtz}[1]{\mathbb{P}_{\theta_0} \left(#1\right)}
\newcommand{\cat}[1]{\text{Cat}\left(#1\right)}
\newcommand{\dir}[1]{\text{Dir}\left(#1\right)}
\newcommand{\ber}[1]{\text{Ber}\left(#1\right)}
\newcommand{\bet}[1]{\text{Beta}\left(#1\right)}
\newcommand{\dete}[1]{\text{det} #1 }
\newcommand{\bin}[1]{\text{Bin}\left(#1\right)}
\newcommand{\mul}[1]{\text{Mult}\left(#1\right)}
\newcommand{\uni}[1]{\text{Uniforme}\left(#1\right)}
\newcommand{\poi}[1]{\text{Poisson}\left(#1\right)}
\newcommand{\expo}[1]{\exp\left(#1\right)}
\newcommand{\loga}[1]{\log\left(#1\right)}
\newcommand{\KL}[2]{\text{KL}\left(#1\middle\|#2\right)}

%conjuntos
\def\R{{\mathbb R}}

%simbolos
\def\ee{{\text{ee}}}
\def\cP{{\mathcal P}}
\def\cA{{\mathcal A}}
\def\cB{{\mathcal B}}
\def\cD{{\mathcal D}}
\def\tq{{\text{ t.q. }}}
\def\cX{{\mathcal X}}
\def\cY{{\mathcal Y}}
\def\cN{{\mathcal N}}
\def\N{{\mathbb N}}
\def\cT{{\mathcal T}}
\def\thetaMV{{\theta_\text{MV}}}
\def\thetahat{{\hat\theta}}
\def\xb{{ \bar{x}}}
\def\gh{{ \hat{g}}}
\def\ghX{{ \hat{g}(X)}}
\def\ghx{{ \hat{g}(x)}}



\def\d{{\text{d}}}
\def\dx{{\text{d}x}}
\def\ind{{\mathbb 1}}
\newcommand{\indep}{\perp \!\!\! \perp}